\section{Funções a valores vetoriais}
Em cursos de Cálculo Diferencial e Integral I, é comum o estudo de funções de uma única variável real a valores reais, ou seja, de funções da forma $f:I\rightarrow \mathbb R$, onde $I\subseteq \mathbb R$ é algum subconjunto de $\mathbb R$, geralmente, um intervalo.
Nesta seção, estudaremos o estudo de funções cujo domínio ainda é algum subconjunto $I$ de $\mathbb R$, mas cujo contradomínio é algo mais geral: um espaço da forma $\mathbb R^n$.

\begin{definition}
    Uma \textbf{função a valores vetoriais} é uma função $f:I\rightarrow \mathbb R^n$, onde $I\subseteq \mathbb R$.
\end{definition}

Interpretações especiais são possíveis quando $n\in \{1, 2, 3\}$.
\begin{itemize}
    \item Se $n=1$, então $f$ é uma função a valores reais, como as estudadas em cursos de Cálculo Diferencial e Integral I.
    \item Se $n=2$, então $f$ é uma função que assume valores em um plano cartesiano.
    \item Se $n=3$, então $f$ é uma função que assume valores em um espaço euclideano tridimensional.
\end{itemize}

No caso $n=2$ e $n=3$, a \emph{imagem} de $f$ adquire uma interpretação geométrica interessante.
Vejamos alguns exemplos.

\begin{example}
    Considere a função $f:\mathbb R\rightarrow \mathbb R^2$ dada por
    \[
        f(t) = (t, t^2).
    \]
    A imagem de $f$ é o conjunto $\{(t, t^2):t\in \mathbb R\}$. Quando um par $(x, y)$ está nesse conjunto?

    Ora, é quando existe $t \in \mathbb R$ tal que $x=t$ e $y=t^2$.
    Ou seja, quando $y=x^2$.
    Logo, a imagem de $f$ é o conjunto $\{(x, y)\in \mathbb R^2:y=x^2\}$, que é uma parábola no plano cartesiano.
\end{example}

\begin{example}
    Considere a função $\gamma:[0, 2\pi[\rightarrow \mathbb R^2$ dada por
    \[
        \gamma(t) = (\cos (t), \sin(t)).
    \]
    A imagem de $\gamma$ é o conjunto $\{(\cos (t), \sin(t)):t\in [0, 2\pi[\}$. Quando um par $(x, y)$ está nesse conjunto?

    Ora, é quando existe $t \in \mathbb R$ tal que $x=\cos(t)$ e $y=\sin(t)$.
    Devemos ter, então que $x^2+y^2=1$.
    Reciprocamente, se $x^2+y^2=1$, então $(x, y)$ está na circunferência de raio $1$, e, portanto, $(x, y)$ é da forma $(\cos(t), \sin(t))$ para algum $t\in [0, 2\pi[$.
\end{example}
\begin{example}
    Considere a função $u:[0, 2\pi[\rightarrow \mathbb R^3$ dada por
    \[
        u(t) = (\cos(t), \sin(t), 2).
    \]

    A imagem de $u$ é a circunferência no plano $z=2$ com raio $1$, centrada no eixo $z$.
\end{example}

\begin{example}
    Considere a função $u:[0, 2\pi[\rightarrow \mathbb R^2$ dada por
    \[
        u(t) = (t, -t, t)=t(1, -1, 1).
    \]
    A imagem de $u$ é o conjunto dos pontos $x, y, z$ para os quais existe $t$ tal que $x=t$, $y=-t$ e $z=t$, o que ocorre se, e somente se, $y=-x$ e $z=x$.
    As equações $y=-x$ e $z=x$ representam planos no espaço tridimensional, e, portanto, a imagem de $u$ é a reta formada pela interseção desses dois planos.
\end{example}


\begin{example}
    Considere a função $u:[0, 2\pi[\rightarrow \mathbb R^3$ dada por
    \[
        u(t) = (2\cos(t), \sin(t)).
    \]

    A imagem de $u$ é o conjunto dos pontos $x, y$ para os quais existe $t$ tais que $x=2\cos(t)$ e $y=\sin(t)$.
    Ou seja, se existe $t$ tal que $\frac{x}{2}=\cos(t)$ e $y=\sin(t)$.
    Ou seja, se, e somente se $\frac{x^2}{4}+y^2=1$.

    A equação $\frac{x^2}{4}+y^2=1$ representa uma elipse no plano cartesiano, com centro na origem.
\end{example}

Podemos pensar de forma inversa: dado um conjunto de pontos, existe uma função cuja imagem é exatamente esse conjunto?
\begin{definition}
    Seja $f:I\rightarrow \mathbb R^n$ uma função a valores vetoriais.
    Dizemos que $f$ \textbf{parametriza} o conjunto $A\subseteq \mathbb R^n$ a imagem de $f$ é $A$.
\end{definition}
\begin{example}
    A função $f(t)=(t, t^2)$ parametriza a parábola $y=x^2$ no plano cartesiano, ou seja, o conjunto $\{(x, y) \in \mathbb R^2: y=x^2\}$.
\end{example}

\begin{example}
    Vamos encontrar uma função a valores reais que parametriza $\{(x, y)\in \mathbb R^2:x^2+y^2=4\}$.
    Notemos que $(x, y)$ está nesse conjunto se, e somente se, $x^2+y^2=4$, o que ocorre se, e somente se $(\frac{x}{2})^2+(\frac{y}{2})^2=1$, o que ocorre se, e somente se existe $t\in [0, 2\pi[$ tal que $\frac{x}{2}=\cos(t)$ e $\frac{y}{2}=\sin(t)$.
    Logo, podemos definir a função $f:[0, 2\pi[\rightarrow \mathbb R^2$ dada por
    \[
        f(t) = (2\cos(t), 2\sin(t)).
    \]
    A imagem de $f$ é exatamente o conjunto $\{(x, y)\in \mathbb R^2:x^2+y^2=4\}$, ou seja, $f$ parametriza esse conjunto.
\end{example}

\begin{example}
    Vamos encontrar uma função a valores reais que parametriza $\{(x, y)\in \mathbb R^2:\frac{x^2}{9}+\frac{y^2}{4}=1\}$.
    Notemos que $(x, y)$ está nesse conjunto se, e somente se, $\frac{x^2}{9}+\frac{y^2}{4}=1$, o que ocorre se, e somente se existe $t\in [0, 2\pi[$ tal que $\frac{x}{3}=\cos(t)$ e $\frac{y}{2}=\sin(t)$.
    Logo, podemos definir a função $f:[0, 2\pi[\rightarrow \mathbb R^2$ dada por
    \[
        f(t) = (3\cos(t), 2\sin(t)).
    \]
    A imagem de $f$ é exatamente o conjunto $\{(x, y)\in \mathbb R^2:\frac{x^2}{9}+\frac{y^2}{4}=1\}$, ou seja, $f$ parametriza esse conjunto.
\end{example}

É tentador chamar, então, uma função $f$ de uma variável real a valores vetoriais de algo como \emph{curva parametrizada}. Porém, um problema vem rapidamente a tona: existem funções a valores reais extremamente mal comportadas, que não se parecem com nossa noção intuitiva de curva.

\begin{example}
    Considere a função $f:\mathbb R\rightarrow \mathbb R^2$ dada por
    \[
        f(t) = \begin{cases}
            (t, 0) & \text{se } t\in \mathbb Q,\\
            (0, t) & \text{se } t\in \mathbb R\setminus \mathbb Q.
        \end{cases}
    \]
\end{example}
\begin{example}
    A seguinte função $f:\mathbb R\rightarrow \mathbb R^2$ parametriza a reta $y=x$:

    \[
        f(t) = (t, t).
    \]

    Porém, a seguinte função também parametriza a reta $y=x$:
    \[
        f(t) = \begin{cases}
            (t, t) & \text{se } t\in [n, n+1[ \text{ para algum } n\in \mathbb Z \text{ ímpar},\\
            (t+2, t+2) & \text{se } t\in [n, n+1[ \text{ para algum } n\in \mathbb Z \text{ par},\\
        \end{cases}
    \]
\end{example}

\begin{example}
    Considere a função $f:\mathbb R\rightarrow \mathbb R^2$ dada por
    \[
        f(t) = \begin{cases}
            (t, \sin(1/t)) & \text{se } t \neq 0,\\
            (0, 0) & \text{se } t=0.
        \end{cases}
    \]
\end{example}