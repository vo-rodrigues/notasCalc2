\section{Continuidade via funções coordenadas}
Se $A$ é qualquer conjunto e $f: A \to \mathbb R^m$ é uma função, então para cada $a \in A$, $f(a)$, sendo um elemento de $\mathbb R^m$, é um ponto $(y_1, \dots, y_m)$.
As coordenadas $y_i$ são denotadas por $f_i(a)$.
Assim, $f(a) = (f_1(a), \dots, f_m(a))$.

\begin{definition}
    Seja $f: A \to \mathbb R^m$ uma função.
    As funções coordenadas de $f$ são as únicas funções $f_i: A \to \mathbb R$ tal que, para todo $a \in A$, temos $f(a) = (f_1(a), \dots, f_m(a))$.
\end{definition}

\begin{proposition}
    Seja $A\subseteq \mathbb R^n$ um conjunto e $f: A \to \mathbb R^m$ uma função.

    Então, $f$ é contínua em $p \in A$ se, e somente se, todas as funções coordenadas $f_i$ são contínuas em $p$.
\end{proposition}

\begin{proof}
    Primeiro, suponha que $f$ é contínua em $p$.
    Vejamos que cada função coordenada $f_i$ é contínua em $p$.
    
    Para isso, fixamos um $i \in \{1, \dots, m\}$ e consideramos $\epsilon>0$.

    Como $f$ é contínua em $p$, existe $\delta>0$ tal que, para todo $x \in A$, se $d(x, p) < \delta$, então $d(f(x), f(p)) < \epsilon$.

    Veremos que o mesmo $\delta$ funciona para $f_i$.
    De fato, se $d(x, p) < \delta$, então temos:

    \begin{align*}
        |f_i(x) - f_i(p)| &= \sqrt{(f_i(x) - f_i(p))^2} \\
        &= \sqrt{(f_1(x) - f_1(p))^2 + \cdots + (f_m(x) - f_m(p))^2} \\
        &= \|f(x) - f(p)\| = d(f(x), f(p)) < \epsilon.
    \end{align*}

    Portanto, $f_i$ é contínua em $p$.

    Reciprocamente, suponha que para todo $i$, $f_i$ é contínua em $p$.
    Fixamos $\epsilon>0$ e, para cada $i$, existe $\delta_i>0$ tal que, se $d(x, p) < \delta_i$, então $|f_i(x) - f_i(p)| < \epsilon$.

    Seja $\delta = \min\{\delta_1, \ldots, \delta_m\}$. Então, se $d(x, p) < \delta$, temos que:

    \begin{align*}
        d(f(x), f(p)) &= \sqrt{(f_1(x) - f_1(p))^2 + \cdots + (f_m(x) - f_m(p))^2} \\
        &< \sqrt{\epsilon^2+\dots+\epsilon^2}= \sqrt{m}\epsilon.
    \end{align*}

    Assim, se de partida, ao invés de $\epsilon$, tomarmos $\delta_1, \dots, \delta_n$ que funcione para $\epsilon'=\frac{\epsilon}{\sqrt{m}}$, concluiremos que para todo $x \in A$ com $d(x, p) < \delta$, temos que $d(f(x), f(p)) < \epsilon'$.
\end{proof}